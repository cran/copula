\documentclass[ps,autumn,slideColor,colorBG]{prosper}

\usepackage[latin1]{inputenc}
\usepackage{pstricks,pst-node,pst-text,pst-3d}
\usepackage{amsmath}
% Definition of new colors
\newrgbcolor{LemonChiffon}{1. 0.98 0.8}
\newrgbcolor{LightBlue}{0.68 0.85 0.9}

% > BEGIN OF OVERLAPPED COLORS
% Code below devised by Denis Girou (CNRS/IDRIS - France, Denis.Girou@idris.fr)
\newrgbcolor{LemonChiffon}{1. 0.98 0.8}
\newrgbcolor{LightBlue}{0.68 0.85 0.9}
\makeatletter
\newdimen\pst@dimz

% Draw two overlapped surfaces, with computation of the mixed color for
% the intersection of the surfaces 
% #1=first  surface, #2=color of first  surface,
% #3=second surface, #4=color of second surface
\def\ColoredOverlappedSurfaces#1#2#3#4{%
\psset{fillstyle=solid}
% Decode the three components of the first RGB color
\DecodeRGBFirstColor{\csname color@#2\endcsname}%
\psset{fillcolor=#2}
% Draw first surface
#1
% Decode the three components of the second RGB color
\DecodeRGBSecondColor{\csname color@#4\endcsname}%
% Compute the mixed color
\BuildMixedColor
% Draw second surface
\psclip{\psset{fillcolor=#4}#3}
\psset{fillcolor=MixedColor}
% Redraw overlapped surface in the mixed color
#1
\endpsclip}

% Get the three components of the first color
\def\DecodeRGBFirstColor#1{%
\pst@expandafter\pst@getnumiii{#1} {} {} {} {}\@nil
\edef\pst@FirstColorR{\pst@tempg}%
\edef\pst@FirstColorG{\pst@temph}%
\edef\pst@FirstColorB{\pst@tempi}%
%\typeout{Color 1=\pst@tempg,\pst@temph,\pst@tempi}% Debug
}

% Get the three components of the second color
\def\DecodeRGBSecondColor#1{%
\pst@expandafter\pst@getnumiii{#1} {} {} {} {}\@nil
\edef\pst@SecondColorR{\pst@tempg}%
\edef\pst@SecondColorG{\pst@temph}%
\edef\pst@SecondColorB{\pst@tempi}%
%\typeout{Color 2=\pst@tempg,\pst@temph,\pst@tempi}% Debug
}

% Build the mixed RBG color (by means of each three components)
\def\BuildMixedColor{%
% Resulting R component
\pst@dimz=\pst@FirstColorR pt
\advance\pst@dimz\pst@SecondColorR pt
\divide\pst@dimz\tw@
\pst@dimtonum{\pst@dimz}{\pst@MixedColorR}%
% Resulting G component
\pst@dimz=\pst@FirstColorG pt
\advance\pst@dimz\pst@SecondColorG pt
\divide\pst@dimz\tw@
\pst@dimtonum{\pst@dimz}{\pst@MixedColorG}%
% Resulting B component
\pst@dimz=\pst@FirstColorB pt
\advance\pst@dimz\pst@SecondColorB pt
\divide\pst@dimz\tw@
\pst@dimtonum{\pst@dimz}{\pst@MixedColorB}%
% Definition of the mixed color MixedColor
\newrgbcolor{MixedColor}{%
\pst@MixedColorR\space \pst@MixedColorG\space \pst@MixedColorB}
%\typeout{Mixed color=\csname color@MixedColor\endcsname}% Debug
}
\makeatother
% < END OF OVERLAPPED COLORS

\title{A small tour of Prosper facilities}
\subtitle{\LaTeX\ presentations made easy}
\author{\href{http://prosper.sourceforge.net/}{{\green Fr�d�ric Goualard}}}
\institution{%
  Centrum voor Wiskunde en Informatica\\
  The Netherlands}



\begin{document}
\maketitle

%---------------------------------------------------------------------- SLIDE -
\begin{slide}{Introduction}
\begin{itemize}
\item If you click on my name in the previous page, you should be
  directed to the Prosper homepage, provided your Acrobat Reader has been
  properly configured.
\item Press on \texttt{CTRL-L} to go to/leave full screen view.
 
\item Curious? Want to go directly to the last page? Push
  \hyperlink{LAST}{{\green here}}.\hypertarget{SECOND}{ }
\end{itemize}
\end{slide}
%------------------------------------------------------------------------------

%---------------------------------------------------------------------- SLIDE -
\overlays{7}{%
\begin{slide}{Transitions}
\texttt{Prosper} offers seven transitions 
between slides:
\begin{itemstep}
\item Split;
\item Blinds;
\item Box;
\item Wipe;
\item Dissolve;
\item Glitter;
\item Replace.
\end{itemstep}
\end{slide}
}
%------------------------------------------------------------------------------


%---------------------------------------------------------------------- SLIDE -
\overlays{2}{%
\begin{slide}{Diagrams}
A small diagram with some few lines of \LaTeX.
\onlySlide{2}{%
  Since the diagram and the text are at the same level, there is no
  difficulty to add some link from one to \rnode{LIEN}{another}.}%

\vspace{0.4cm}
{\tiny
\begin{equation*}
\setlength{\arraycolsep}{1cm}
\def\tX{\tilde{\tilde{X}}}
\begin{array}{cc}
        (X-A,N-A)\rnode{a}{} & \rnode{b}{}(\tX,a)\\[1.5cm]
        (X,N)\rnode{c}{} & \rnode{d}{}(\tX,N)\\[1.5cm]
\end{array}
\psset{nodesep=5pt,arrows=->}
\onlySlide*{2}{\nccurve[linecolor=white,angleA=270,angleB=180]{LIEN}{d}}%
\ncline[linecolor=white]{a}{b}\Aput{a}
\ncline[linecolor=white]{a}{c}\Bput{r}
\ncline[linecolor=white,linestyle=dashed]{c}{d}\Bput{b}
\ncline[linecolor=white]{b}{d}\Bput{s}
\end{equation*}}
\end{slide}
}
%------------------------------------------------------------------------------


%---------------------------------------------------------------------- SLIDE -
\overlays{2}{%
\begin{slide}[Dissolve]{A small \emph{clipping} effect}
\small
Any practical use for this?

\begin{psclip}{\psellipse[linestyle=none]%
                (4.9,-1.7)(4,1.6)}
\begin{center}
\parbox{7cm}{%
\green
Ce n'�tait pas une petite gare de province, mais une porte d�rob�e.
Elle donnait en apparence sur la campagne. Sous l'\oe{}il d'un 
contr\^oleur paisible on gagnait une route blanche sans myst�re,
un ruisseau, des �glantines. Le chef de gare soignait des roses, 
l'homme d'�quipe feignait de pousser un chariot vide. Sous ces 
d�guisements, veillaient trois gardiens d'un monde secret.}
\end{center}
\end{psclip}

\vspace*{-2cm}
\OnlySlide{2}%
\pstextpath{\psccurve[linestyle=none](.5,0)(3.5,1)(3.5,0)(.5,1)}{\green And there are so many other funny effects\dots}
\end{slide}
}
%------------------------------------------------------------------------------


%---------------------------------------------------------------------- SLIDE -
\overlays{3}{%
\begin{slide}{Householder formula}
\small
The Householder formula below lets you compute $f^{-1}(x)$ for an arbitrary
$f$.
{\scriptsize
\begin{equation}\label{Householder}
x_{k+1}\mapsto \Phi_n(x_k)=x_k+(n-1)
\frac{\bigl(\frac{1}{f(x_k)}\bigr)^{n-2}}{\bigl(\frac{1}{f(x_k)}\bigr)^{n-1}}+
f(x_k)^{n+1}%
\fromSlide*{2}{\rnode{NA}{\pscirclebox[linecolor=red]{\psi}}}
\onlySlide*{1}{\rnode{NA}{\pscirclebox[linecolor=red,linestyle=none]{\psi}}}
\end{equation}}

\FromSlide{2}%
where $n\geq 2$ and \rnode{NB}{$\psi$} is an arbitrary function.
\fromSlide*{3}{\nccurve[linecolor=red,angleA=90,angleB=270]{->}{NB}{NA}}

\OnlySlide{3}%
Formula~\eqref{Householder} gives an iteration of order $n$ converging
towards $x_*$ such that: $f(x_*)=0$.
\end{slide}
}
%------------------------------------------------------------------------------

%---------------------------------------------------------------------- SLIDE -
\overlays{2}{%
  \begin{slide}[Glitter]{Overlaps of colors}
    Intersection of sets. First the yellow one\dots\onlySlide{2}{Then
      the blue one.}%
    \rput(4,-2){%
      \onlySlide{1}{%
        \psellipse[fillstyle=solid,fillcolor=LemonChiffon](0.3,-0.7)(1.5,1)}%
      \onlySlide{2}{%
        \ColoredOverlappedSurfaces{\pscircle{1}}{LightBlue}%
                         {\psellipse[fillstyle=solid,%
                           fillcolor=LemonChiffon](0.3,-0.7)(1.5,1)}{%
                           LemonChiffon}}}%
    \OnlySlide{2}%
    {\green Remember how to do that with MS PowerPoint?}
  \end{slide}
}
%------------------------------------------------------------------------------

%---------------------------------------------------------------------- SLIDE -
\begin{slide}{Last slide}
  This is the \hypertarget{LAST}{last} slide. Do you want to go to the 
  \hyperlink{SECOND}{{\green second one}}?
\end{slide}
%---------------------------------------------------------------------- SLIDE -


\end{document}

%%% Local Variables: 
%%% mode: latex
%%% TeX-master: t
%%% End: 
